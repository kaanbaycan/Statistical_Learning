\documentclass{article}
\usepackage{graphicx} % Required for inserting images

\title{Exercises}
\author{Yakup Kaan Baycan}
\date{July 2023}

\begin{document}

\maketitle

\section{2.4 Statistical Learning Exercises}
\subsection{Conceptual}

\textbf{(a) The sample size n is extremely large, and the number of predictors p is small.} \\

Since n is large, and p is small we tend to use a \textbf{flexible model}. Although a flexible model will not perfectly fit to data it is better compared to inflexible models. Hence, an inflexible model will require more parameters to fit, and as the model gets more complex and flexible we may face \textbf{over-fitting problem} \\

\textbf{(b) The number of predictors p is extremely large, and the number
of observations n is small.}\\

In this case, we may go with an \textbf{inflexible model} since we do not have enough data to train. An inflexible model behaves better with a small n and reduces the risk of over-fitting.\\

\textbf{(c) The relationship between the predictors and response is highly
non-linear.}\\

If we are aware that the response variable is non-linear, then using a \textbf{flexible model} will be more accurate since it performs well when there is non-linearity. Hence, an inflexible model will have a higher bias which will result in a worse score.\\

\textbf{(d) The variance of the error terms, i.e. $\sigma^2$ = $Var(\epsilon)$, is extremely high.}\\

In such case, we may use the formula MSE = $Bias(\hat{y})^2 + Var(\hat{y})$. If the variance of the error terms is extremely high, the flexible model may overfit easily and fail on the test set. So we should go with an \textbf{inflexible model} which may seem less accurate in the training set but will perform better than a flexible one on new data.\\


\end{document}
